En esta fase se definieron las partes interesadas del proyecto, los roles de Scrum que cada miembro del equipo cumplirá y además, se realizó la planificación para el desarrollo del \textit{middleware}.

\subsection{Stakeholders}
Para poder cubrir las necesidades de este componente, en la Tabla \ref{tab:stakeholders} se identificaron los siguiente stakeholders y el papel que cumplirá cada uno de ellos:

\begin{table}[H]
    \centering
    \caption{Partes interesadas del proyecto}
    \label{tab:stakeholders}
    \begin{tabular}{|l|l|p{0.6\textwidth}|}
    \hline
    \rowcolor{gray!50}
    \multicolumn{1}{|c|}{\textbf{ID}} & \multicolumn{1}{c|}{\textbf{Stakeholder}} & \multicolumn{1}{c|}{\textbf{Descripción}} \\
    \hline
    S1 & Desarrollador (Usuario) & Persona que interactuará con el traductor de consultas SQL a un lenguaje de consulta de grafos. \\
    \hline
    S2 & Docente & Persona encargada de supervisar el cumplimiento de los objetivos y necesidades del componente.\\
    \hline
    S3 & Desarrollador & Persona responsable del diseño y construcción del \textit{middleware} asegurando que se cumplan con todos los requisitos definidos. \\
    \hline
    \end{tabular}
\end{table}


\subsection{Planificación del proyecto}
Una vez identificados los stakeholders, se estableció la planificación del proyecto donde se contempló 358 horas de trabajo dentro de un periodo de 5 meses para el desarrollo del producto. Aquí se definieron actividades que cubren el análisis de requerimientos, diseño, implementación y pruebas que se realizarán al sistema.

En la Figura \ref{fig:planificacion} se indica la lista de actividades consideradas con el número de horas asignadas a cada una de ellas.

\begin{figure}[ht]
    \centering
    \includegraphics[width=0.8\linewidth]{../02Figures/02Chapter/C2.ProjectSchedule.png}
    \caption{Planificación del proyecto.}
    \label{fig:planificacion}
\end{figure}

A lo largo del proyecto se llevaron a cabo reuniones semanales con el Ph.D. Víctor Velepucha, director del proyecto de integración curricular, con la finalidad de obtener una basta retroalimentación que permita la correcta construcción del \textit{middleware}.

\subsection{Roles de Scrum}
Dado que se utilizó el marco de trabajo Scrum para una entrega de valor de manera ágil, es indispensable determinar los roles que cada miembro cumplirá dentro del equipo de trabajo. En la Tabla \ref{tab:rolesProyecto} se detalla la asignación de roles para el proyecto:
\begin{table}[H]
    \centering
    \caption{Roles de Scrum}
    \label{tab:rolesProyecto}
    \begin{tabular}{|l|p{0.7\textwidth}|}
    \hline
    \rowcolor{gray!50}
    \multicolumn{1}{|c|}{\textbf{Rol}} & \multicolumn{1}{c|}{\textbf{Persona(s) Asignada(s)}} \\
    \hline
    Product Owner & Víctor Velepucha \\
    \hline
    Scrum Master & Víctor Velepucha \\
    \hline
    Equipo de Desarrollo & Andrés Cantuña, Sebastián Sánchez y René Simbaña \\
    \hline
    \end{tabular}
\end{table} 