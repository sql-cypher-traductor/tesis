Para esta fase se redactaron las historias de usuario que cubran las necesidades del negocio y se realizó la selección del stack tecnológico que permita la gestión y el desarrollo eficiente del sistema.


\subsection{Herramientas}
La Tabla \ref{tab:stackTecnologico} indica las herramientas elegidas para llevar a cabo el diseño y el desarrollo del sistema. Estas herramientas fueron elegidas en base a la versatilidad y la fácil integración entre sí.

\begin{longtable}{|l|p{0.8\textwidth}|}
    \caption{Stack tecnológico.}
    \label{tab:stackTecnologico} \\
    \hline
    \rowcolor{gray!50}
    \multicolumn{1}{|c|}{\textbf{Herramienta}} & \multicolumn{1}{c|}{\textbf{Descripción}} \\
    \hline
    \endfirsthead

    \hline
    \rowcolor{gray!50}
    \multicolumn{1}{|c|}{\textbf{Herramienta}} & \multicolumn{1}{c|}{\textbf{Descripción}} \\
    \hline
    \endhead

    \multicolumn{2}{r}{\textit{Continúa en la siguiente página...}} \\
    \endfoot

    \endlastfoot

    Figma & Utilizado para el diseño de las interfaces de usuario del sistema. \\
    \hline
    Visual Studio Code & Editor de código utilizado para el desarrollo del sistema debido a su versatilidad y compatibilidad con múltiples lenguajes y frameworks.\\
    \hline
    Postman & Herramienta que permitirá evaluar los endpoints del \textit{middleware} para asegurar su correcta integración con otros sistemas. \\
    \hline
    Git y GitHub & Se utilizarán tanto para el control de versiones como para el almacenamiento del código del proyecto. \\
    \hline
\end{longtable}

\subsection{Configuración del Proyecto}

\subsubsection{Historias de Usuario}
Una vez precisado el desafío de negocio, se definieron las Historias de Usuario (HU) en base a los requisitos que el \textit{middleware} debe cubrir. Estas HU fueron agrupadas en épicas de usuario tal y como se aprecia en la Tabla \ref{tab:epicasDeUsuario}, que incluye el nombre de la épica junto a una descripción y las HU correspondientes.

% Modificar la tabla de épicas
\begin{table}[H]
    \centering
    \caption{Épicas del proyecto.}
    \label{tab:epicasDeUsuario}
    \begin{tabular}{|l|p{0.25\textwidth}|p{0.4\textwidth}|p{0.2\textwidth}|}
    \hline
    \rowcolor{gray!50}
    \multicolumn{1}{|c|}{\textbf{N°}} & \multicolumn{1}{c|}{\textbf{Épica}} & \multicolumn{1}{c|}{\textbf{Descripción}} & \multicolumn{1}{c|}{\textbf{Historias de Usuario}} \\
    \hline
    1 & Traducción de consultas & Facilita el análisis y traducción de sentencias SQL a su equivalente en lenguaje Cypher & HU01, HU02 \\
    \hline
    2 & Conexion y ejecución de consultas & Permite al usuario conectarse a una base de datos de grafos específica y ejecutar consultas traducidas de forma directa. & HU04, HU05 \\
    \hline
    3 & Gestión de utilidades y retroalimentación & Permite al usuario el manejo de consultas anteriores y resultados según sea su necesidad, además de proporcionar retroalimentación que le brinde una mejor experiencia de uso. & HU03, HU06 y HU07 \\
    \hline
    \end{tabular}
\end{table}



A continuación, la Tabla \ref{tab:hu01} presenta a través de la HU01, a manera de ejemplo, el formato elegido para las historias de usuario que se encuentran dentro del Anexo \ref{Annexes:AnexoI}. Este formato contempla campos importantes como la prioridad, estimación, historia de usuario como tal o los criterios de aceptación.

\begin{longtable}{|p{0.5\textwidth}|p{0.5\textwidth}|}
    \caption{Historia de Usuario 01}
    \label{tab:hu01} \\
    
    \hline
    \endfirsthead
    \endhead
    \multicolumn{2}{r}{\textit{Continúa en la siguiente página...}} \\
    \endfoot
    \endlastfoot

    \textbf{ID:} HU01 & \textbf{Usuario:} Desarrollador \\
    \hline
    \multicolumn{2}{|l|}{\textbf{Título:} Traducción de consultas básicas} \\
    \hline
    \textbf{Prioridad en negocio:} Alta & \textbf{Riesgo en desarrollo:} Alto \\
    \hline
    \textbf{Estimación:} 8 & \textbf{Iteración asignada:} 1 \\
    \hline
    \multicolumn{2}{|l|}{\textbf{Historia de Usuario:}} \\
    \hline
    \multicolumn{2}{|p{1\textwidth}|}{\textbf{Como} desarrollador, \newline \textbf{Quiero} traducir sentencias SELECT, CREATE, UPDATE o DELETE simples a lenguaje Cypher, \newline \textbf{Para} interactuar con bases de datos orientadas a grafos sin tener que aprender un nuevo lenguaje de consulta.} \\
    \hline
    \multicolumn{2}{|l|}{\textbf{Criterios de aceptación:}} \\
    \hline
    \multicolumn{2}{|p{1\textwidth}|}{\begin{itemize}
        \item El usuario debe agregar una sentencia SQL válida para ser traducida.
        \item Las sentencias soporta cláusulas, operadores y funciones básicas.
        \item Al ingresar una sentencia válida, el sistema devolverá su equivalente en lenguaje Cypher.
        \item Las consultas traducidas serán agregadas al historial de consultas.
    \end{itemize}} \\
    \hline
    \multicolumn{2}{|l|}{\textbf{Observaciones:}} \\
    \hline
    \multicolumn{2}{|p{1\textwidth}|}{\begin{itemize}
        \item Verificar que la consulta a ser traducida no contenga errores de sintaxis.
        \item La traducción a Cypher aparecerá junto a la sentencia SQL correspondiente.
    \end{itemize}} \\
    \hline
\end{longtable}

\subsubsection{Product Backlog}
Para la construcción del Product Backlog, cada una de las Historias de Usuario contempladas serán desglosadas en tareas que nos permitirá entregar valor un incremento de forma rápida y continua, tal y como lo indica el marco de trabajo Scrum. Para ello, se utilizará la técnica \textbf{Planning Poker} que ayudará a evaluar la prioridad y complejidad que conlleva cada tarea.

En la Tabla \ref{tab:prioridad} y en la Tabla \ref{tab:complejidad} se muestran las escalas de valoración de prioridad y complejidad, establecidas mediante la serie de Fibonacci.

\begin{table}[H]
    \centering
    \caption{Valoración de prioridad.}
    \label{tab:prioridad}
    \begin{tabular}{|c|c|}
    \hline
    \rowcolor{gray!50}
    \textbf{Valor} & \textbf{Prioridad} \\
    \hline
    1 & Baja \\
    \hline
    2 & Media \\
    \hline
    3 & Alta \\
    \hline
    \end{tabular}
\end{table}

\begin{table}[H]
    \centering
    \caption{Valoración de complejidad.}
    \label{tab:complejidad}
    \begin{tabular}{|c|c|}
    \hline
    \rowcolor{gray!50}
    \textbf{Valor} & \textbf{Complejidad} \\
    \hline
    1 & Muy baja \\
    \hline
    2 & Baja \\
    \hline
    3 & Media \\
    \hline
    5 & Alta \\
    \hline
    8 & Muy alta \\
    \hline
    \end{tabular}
\end{table}


Con los criterios claros y en colaboración con el Scrum Team, se construyó el Product Backlog que se muestra en la Tabla \ref{tab:productBacklog}. Aquí se indica la tarea a realizar, su complejidad y prioridad, así como la historia de usuario a la cual corresponde.


\begin{longtable}{|c|c|p{0.35\textwidth}|c|c|}
    \caption{Product backlog del proyecto.}
    \label{tab:productBacklog} \\
    \hline
    \rowcolor{gray!50}
    \multicolumn{1}{|c|}{\textbf{HU}} & \multicolumn{1}{|c|}{\textbf{Código de tarea}} & \multicolumn{1}{|c|}{\textbf{Tarea}} & \multicolumn{1}{c|}{\textbf{Complejidad}} & \multicolumn{1}{c|}{\textbf{Prioridad}}  \\
    \hline
    \endfirsthead
    
    \hline
    \rowcolor{gray!50}
    \multicolumn{1}{|c|}{\textbf{HU}} & \multicolumn{1}{|c|}{\textbf{Código de tarea}} & \multicolumn{1}{|c|}{\textbf{Tarea}} & \multicolumn{1}{c|}{\textbf{Complejidad}} & \multicolumn{1}{c|}{\textbf{Prioridad}}  \\
    \hline
    \endhead
    
    \multicolumn{5}{|r|}{\textit{Continúa en la siguiente página...}} \\
    \endfoot
    
    \endlastfoot
    
    \multirow{4}{*}{HU01} & T1.1 & Desarrollar la lógica de parseo para identificar el tipo de sentencia SQL y mapeo para Cypher. & 8 & 3 \\
    \cline{2-5}
     & T1.2 & Implementar un endpoint para la traducción de SQL a Cypher. & 2 & 3 \\
    \cline{2-5}
     & T1.3 & Diseñar la pantalla para traducción. & 3 & 3 \\
    \cline{2-5}
     & T1.4 & Desarrollar de la vista para traducción. & 2 & 3 \\
    \hline
    \multirow{3}{*}{HU02} & T2.1 & Desarrollar la lógica para detección de JOINs y claves foráneas. & 8 & 2 \\
    \cline{2-5}
     & T2.2 & Implementar el mapeo de claves foráneas a relaciones de grafos. & 5 & 2 \\
    \cline{2-5}
     & T2.3 & Ajustar la visualización de consultas que incluyan JOINs. & 1 & 1 \\
    \hline
    \multirow{3}{*}{HU03} & T3.1 & Diseñar el componente para visualización de errores. & 1 & 2 \\
    \cline{2-5}
     & T3.2 & Desarrollar el componente para visualización de errores. & 1 & 2 \\
    \cline{2-5}
     & T3.3 & Implementar el detalle de error y sugerencia de corrección. & 3 & 2 \\
    \hline
    \multirow{4}{*}{HU04} & T4.1 & Desarrollar la lógica para conexión a una base de datos Neo4j. & 5 & 3 \\
    \cline{2-5}
     & T4.2 & Implementar un endpoint para conectarse a la base de datos de grafos. & 3 & 3 \\
    \cline{2-5}
     & T4.3 & Diseñar el formulario de conexión. & 2 & 2 \\
    \cline{2-5}
     & T4.4 & Desarrollar el formulario de conexión. & 2 & 3 \\
    \hline
    \multirow{4}{*}{HU05} & T5.1 & Implementar la lógica para la ejecución de la traducción en la base de datos de grafos. & 5 & 3 \\
    \cline{2-5}
     & T5.2 & Implementar un endpoint para la ejecución de la traducción obtenida en la base de datos de Neo4j. & 3 & 3 \\
    \cline{2-5}
     & T5.3 & Diseñar los resultados de la ejecución. & 1 & 2 \\
    \cline{2-5}
     & T5.4 & Desarrollar del área de resultados con opciones de visualización en formato tabular y JSON. & 3 & 3 \\
    \hline
    \multirow{3}{*}{HU06} & T6.1 & Implementar la lógica para el almacenamiento de consultas. & 2 & 1 \\
    \cline{2-5}
     & T6.2 & Desarrollar la lógica para rellenar los cuadros de texto de traducción al seleccionar una consulta guardada. & 1 & 1 \\
    \cline{2-5}
     & T6.3 & Diseñar el historial de consultas. & 2 & 2 \\
    \cline{2-5}
     & T6.4 & Desarrollar la vista del historial de consultas. & 2 & 3 \\
    \hline
    \multirow{2}{*}{HU07} & T7.1 & Implementar la transformación de JSON a CSV. & 3 & 1 \\
    \cline{2-5}
     & T7.2 & Desarrollar la función de descarga de resultados. & 1 & 1 \\
    \hline
\end{longtable}

\subsubsection{Planificación de Sprints}
Con la finalidad de cumplir con el Product Backlog establecido, se decidió dividir el trabajo en 5 sprints, incluyendo un sprint 0 dedicado a la configuración del entorno de desarrollo, el diseño de arquitectura del sistema y las interfaces de usuario respectivas.

En la Tabla \ref{tab:sprints}, se detalla la planificación de cada sprint, donde se indica la duración en semanas, los objetivos a cumplir y las tareas asignadas a cada uno de ellos.

\begin{table}[H]
    \centering
    \caption{Planificación de Sprints.}
    \label{tab:sprints}
    \begin{tabular}{|c|c|m{0.4\textwidth}|m{0.25\textwidth}|}
    \hline
    \rowcolor{gray!50}
    \multicolumn{1}{|c|}{\textbf{Sprint}} & \multicolumn{1}{c|}{\textbf{Duración}} & \multicolumn{1}{c|}{\textbf{Objetivos}} & \multicolumn{1}{c|}{\textbf{Tareas}} \\
    \hline
    Sprint 0 & 2 semanas & \begin{itemize}
        \item Configurar el entorno de trabajo.
        \item Diseñar la arquitectura del middleware.
        \item Prototipar las interfaces de usuario del sistema.
    \end{itemize} & T1.3, T3.1, T4.3, T5.3, T6.3 \\
    \hline
    Sprint 1 & 3 semanas & \begin{itemize}
        \item Implementar la traducción de consultas SQL simples.
    \end{itemize} & T1.1, T1.2, T1.4 \\
    \hline
    Sprint 2 & 3 semanas & \begin{itemize}
        \item Implementar la traducción de consultas SQL que incluyan relaciones.
    \end{itemize} & T2.1, T2.2, T2.3 \\
    \hline
    Sprint 3 & 3 semanas & \begin{itemize}
        \item Implementar la retroalimentación de errores y sugerencias de corrección.
        \item Implementar la conexión a una base de datos de grafos Neo4j.
    \end{itemize} & T3.2, T3.3, T4.1, T4.2, T4.4 \\
    \hline
    Sprint 4 & 3 semanas & \begin{itemize}
        \item Implementar la ejecución de consultas traducidas en una base de datos Neo4j.
        \item Implementar el historial de consultas.
    \end{itemize} & T5.1, T5.2, T5.4, T6.1, T6.2, T6.4 \\
    \hline
    Sprint 5 & 2 semanas & \begin{itemize}
        \item Implementar la exportación de resultados de ejecución.
    \end{itemize} & T7.1, T7.2 \\
    \hline
    \end{tabular}
\end{table}


\subsection{Sprint 0}
\subsubsection{Sprint 0 Planning}
Dentro del sprint 0 se estableció la arquitectura del sistema y el diseño de las interfaces de usuario así también, se preparó el entorno de trabajo, el cual incluye la creación de los repositorios y la estructura del código tanto para el backend como para el frontend, de acuerdo al stack tecnológico elegido.

\subsubsection{Sprint 0 Backlog}
Este backlog incluye tareas relacionadas al diseño del middleware que nos ayudará como guía y base para los próximos sprints. La Tabla \ref{tab:sprintBacklog0} detalla las tareas del Sprint 0.

\begin{table}[H]
    \centering
    \caption{Sprint 0 Backlog.}
    \label{tab:sprintBacklog0}
    \begin{tabular}{|c|p{0.8\textwidth}|}
    \hline
    \rowcolor{gray!50}
    \textbf{Código} & \textbf{Tarea} \\
    \hline
    T1.3 & Diseñar la pantalla para traducción. \\
    \hline
    T3.1 & Diseñar el componente para visualización de errores. \\
    \hline
    T4.3 & Diseñar el formulario de conexión. \\
    \hline
    T5.3 & Diseñar los resultados de la ejecución. \\
    \hline
    T6.3 & Diseñar el historial de consultas. \\
    \hline
    \end{tabular}
\end{table}


\subsubsection{Ejecución del Sprint 0}
Durante la ejecución de este Sprint, el equipo se centró en establecer los cimientos del proyecto. Las actividades principales se dividieron en el diseño de la experiencia de usuario y la configuración técnica inicial.

En primer lugar, se abordó el diseño de las interfaces de usuario, asegurando que la interacción con el \textit{middleware} fuera intuitiva y eficiente. Se diseñaron las pantallas clave para la traducción de sentencias (T1.3), la visualización de errores (T3.1), el formulario de conexión a la base de datos (T4.3), la presentación de resultados de ejecución (T5.3) y el historial de consultas (T6.3). Estos diseños sirvieron como guía visual para el desarrollo del frontend en los sprints subsiguientes.

Paralelamente, se definió la arquitectura del sistema y se configuró el entorno de desarrollo. Esto incluyó la creación de los repositorios de código, la configuración de las herramientas de control de versiones y la preparación de la estructura base tanto para el \textit{backend} como para el \textit{frontend}, respetando el stack tecnológico seleccionado previamente.

Adicionalmente, se estableció el entorno de pruebas para las futuras APIs. Se creó una colección compartida en \textbf{Postman}, definiendo las variables de entorno globales (como la URL base del servidor de desarrollo) y la estructura de carpetas para organizar las peticiones por módulos, lo que facilitaría las pruebas de integración en los siguientes sprints.

% Se sugiere agregar aquí una figura con los prototipos de las interfaces diseñadas (ej. Figura X: Prototipos de UI del Sprint 0).

\subsubsection{Sprint 0 Review}
Al finalizar el Sprint 0, se realizó la revisión con los interesados para validar los artefactos generados. Se presentaron los prototipos de alta fidelidad de las interfaces de usuario, obteniendo retroalimentación positiva sobre la disposición de los elementos y el flujo de navegación propuesto. Asimismo, se verificó la correcta configuración del entorno de desarrollo y la arquitectura base, confirmando que el equipo contaba con todas las herramientas necesarias para iniciar la codificación de las funcionalidades en el siguiente sprint.

\subsubsection{Sprint 0 Retrospective}
La retrospectiva del Sprint 0 permitió al equipo reflexionar sobre el inicio del proyecto.

\begin{itemize}
    \item \textbf{¿Qué salió bien?} \newline
    La definición temprana de la arquitectura y el stack tecnológico permitió configurar el entorno sin contratiempos mayores. La comunicación fluida durante la fase de diseño facilitó la creación de prototipos que cumplían con las expectativas de los interesados.
    \item \textbf{¿Qué se puede mejorar?} \newline
    Se identificó que la estimación de tiempo para algunas tareas de diseño fue ajustada, por lo que se acordó refinar los criterios de estimación para futuras tareas creativas. Además, se propuso documentar con mayor detalle las decisiones arquitectónicas a medida que se toman.
\end{itemize}
