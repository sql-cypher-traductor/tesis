El alcance del proyecto comprende el desarrollo de un middleware unificado SQL para bases de datos NoSQL, con capacidad para traducir consultas SQL en instrucciones compatibles con bases de datos orientadas a grafos, clave-valor y geoespaciales. Este componente incluye:
\begin{enumerate}
    \item \textbf{Revisión de literatura y análisis de requisitos:} Se realizará una investigación exhaustiva sobre las técnicas y herramientas existentes para la traducción de consultas SQL a NoSQL, así como un análisis de los requisitos necesarios para el desarrollo del middleware.
    \item \textbf{Definición de la gramática SQL:} Utilizando la herramienta adecuada, se definirá y generará la gramática SQL necesaria para el análisis y descomposición de consultas SQL en componentes manejables.
    \item \textbf{Desarrollo del parser SQL:} Implementación del parser SQL que analizará las consultas SQL y generará un árbol de análisis que servirá de base para la traducción a lenguajes de consulta NoSQL.
    \item \textbf{Desarrollo de conectores NoSQL:} Implementación de un conector que traduzca consultas SQL al lenguaje de consulta Cypher. Este conector permitirá la ejecución de consultas traducidas, enfocándose en las relaciones y estructuras interconectadas.
    \item \textbf{Desarrollo de conector:} Implementación de un conector que permita recibir consultas SQL y enviarlas al middleware. El middleware será responsable de traducir estas consultas SQL a una representación intermedia.
    \item \textbf{Interfaz de usuario:} El middleware incluirá una interfaz de usuario que permitirá a los desarrolladores interactuar con el sistema de manera directa. A través de esta interfaz, los usuarios podrán ingresar consultas SQL y recibir los resultados traducidos y ejecutados en el sistema NoSQL correspondiente.
\end{enumerate}
%\lipsum[1-3]

%%%---------- Tabla 1. Plan de Tesis ----------%%%
%\input{03Tables/01Chapter/C1.tableExample.tex}