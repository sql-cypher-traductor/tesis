\abstract{spanish}{Este Trabajo de Integración Curricular (TIC) se centra en el desarrollo de un middleware traductor de sentencias SQL para sistemas de bases de datos orientadas a grafos.
Actualmente, existe una brecha en la integración de aplicaciones tradicionales que manejan bases de datos SQL con tecnologías NoSQL, que hace que los desarrolladores de bases de datos tengan que aprender nuevos lenguajes de consulta, lo cual implica un tiempo adicional y reduce su productividad.
Ante este problema, se propone como solución un sistema que traduzca sentencias SQL en su equivalente a un GQL como Cypher y la ejecución directa de esta traducción en una base de datos orientada a grafos sin la necesidad de ingresar a un GDBMS como Neo4j.
Para una entrega de valor continua e incremental, se optó por el uso del marco de trabajo SCRUM, estableciéndose 6 sprints para el desarrollo del middleware.
El backend fue realizado con FastAPI incluye un parser SQL capaz de analizar la consulta ingresada, un traductor implementado mediante reglas de mapeo y los conectores respectivos tanto para la base de datos SQL como para la de Neo4j.
Por su parte, el frontend fue construido con NextJS, que permitirá a los desarrolladores ingresar las sentencias SQL y visualizar tanto la traducción como los resultados de la ejecución de forma fácil e intuitiva.
Con este trabajo se busca disminuir curva de aprendizaje hacia tecnologías desconocidas, y aumentar la interoperabilidad entre sistemas SQL y NoSQL contribuyendo a una migración más rápida y eficiente.}{Parser SQL, Traductor SQL-NoSQL, Scrum, Neo4j, Cypher}

\abstract{english}{This Curricular Integration Project (CIP) focuses on the development of middleware that translates SQL statements for graph-oriented database systems.
Currently, there is a gap in the integration of traditional applications that handle SQL databases with NoSQL technologies, which means that database developers have to learn new query languages, which takes additional time and reduces their productivity.
To address this problem, the proposed solution is a system that translates SQL statements into their equivalent in a GQL such as Cypher and executes this translation directly in a graph-oriented database without the need to enter a GDBMS such as Neo4j.
For continuous and incremental value delivery, the SCRUM framework was chosen, with six sprints established for the development of the middleware.
The backend was built with FastAPI and includes an SQL parser capable of analyzing the query entered, a translator implemented using mapping rules, and the respective connectors for both the SQL database and the Neo4j database.
The frontend was built with NextJS, which will allow developers to enter SQL statements and view both the translation and the execution results in an easy and intuitive way.
This work seeks to reduce the learning curve for unfamiliar technologies and increase interoperability between SQL and NoSQL systems, contributing to faster and more efficient migration.}{SQL Parser, SQL-NoSQL Translator, Scrum, Neo4j, Cypher}