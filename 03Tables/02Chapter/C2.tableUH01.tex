\begin{longtable}{|p{0.5\textwidth}|p{0.5\textwidth}|}
    \caption{Historia de Usuario 01}
    \label{tab:hu01} \\
    
    \hline
    \endfirsthead
    \endhead
    \multicolumn{2}{r}{\textit{Continúa en la siguiente página...}} \\
    \endfoot
    \endlastfoot

    \textbf{ID:} HU01 & \textbf{Usuario:} Desarrollador \\
    \hline
    \multicolumn{2}{|l|}{\textbf{Título:} Traducción de consultas básicas} \\
    \hline
    \textbf{Prioridad en negocio:} Alta & \textbf{Riesgo en desarrollo:} Alto \\
    \hline
    \textbf{Estimación:} 8 & \textbf{Iteración asignada:} 1 \\
    \hline
    \multicolumn{2}{|l|}{\textbf{Historia de Usuario:}} \\
    \hline
    \multicolumn{2}{|p{1\textwidth}|}{\textbf{Como} desarrollador, \newline \textbf{Quiero} traducir sentencias SELECT, CREATE, UPDATE o DELETE simples a lenguaje Cypher, \newline \textbf{Para} interactuar con bases de datos orientadas a grafos sin tener que aprender un nuevo lenguaje de consulta.} \\
    \hline
    \multicolumn{2}{|l|}{\textbf{Criterios de aceptación:}} \\
    \hline
    \multicolumn{2}{|p{1\textwidth}|}{\begin{itemize}
        \item El usuario debe agregar una sentencia SQL válida para ser traducida.
        \item Las sentencias soporta cláusulas, operadores y funciones básicas.
        \item Al ingresar una sentencia válida, el sistema devolverá su equivalente en lenguaje Cypher.
        \item Las consultas traducidas serán agregadas al historial de consultas.
    \end{itemize}} \\
    \hline
    \multicolumn{2}{|l|}{\textbf{Observaciones:}} \\
    \hline
    \multicolumn{2}{|p{1\textwidth}|}{\begin{itemize}
        \item Verificar que la consulta a ser traducida no contenga errores de sintaxis.
        \item La traducción a Cypher aparecerá junto a la sentencia SQL correspondiente.
    \end{itemize}} \\
    \hline
\end{longtable}