\begin{longtable}{|l|p{0.5\textwidth}|p{0.2\textwidth}|}
    \caption{Eventos de Scrum}
    \label{tab:eventos} \\
    \hline
    \rowcolor{gray!50}
    \multicolumn{1}{|c|}{\textbf{Evento}} & \multicolumn{1}{c|}{\textbf{Descripción}} & \multicolumn{1}{c|}{\textbf{Duración}} \\
    \hline
    \endfirsthead

    \hline
    \rowcolor{gray!50}
    \multicolumn{1}{|c|}{\textbf{Evento}} & \multicolumn{1}{c|}{\textbf{Descripción}} & \multicolumn{1}{c|}{\textbf{Duración}} \\
    \hline
    \endhead

    \multicolumn{3}{r}{\textit{Continúa en la siguiente página...}} \\
    \endfoot

    \endlastfoot


    Sprint & Un ciclo de trabajo de duración fija, en el que se crea un incremento de producto potencialmente entregable. & De 2 a 4 semanas \\
    \hline
    Sprint Planning & La reunión donde el equipo planifica el trabajo a realizar durante el sprint y define su objetivo. & 2 horas por semana de Sprint \\
    \hline
    Daily Scrum & Una reunión diaria de 15 minutos para que el equipo sincronice sus actividades y adapte el plan del día. & 15 minutos \\
    \hline
    Sprint Review & El encuentro donde el equipo inspecciona el incremento de producto con los stakeholders y obtiene retroalimentación. & 1 hora por semana de Sprint \\
    \hline
    Sprint Retrospective & Una reunión para que el equipo inspeccione su proceso de trabajo e identifique mejoras para el próximo sprint. & 45 minutos por semana de Sprint. \\
    \hline

\end{longtable}